
\section{Results (Tips for Good Software)}
\label{sec:results}
\label{sec:tips-software}

There is a lot of literature (for example~\cite{hunt99pragmatic} and
\cite{spolsky04software}) on how to write software. It is not the
intention of this section to replace software engineering
courses. However, in the interests of reproducible
research~\cite{schwab00}, there are a few guidelines to make your
reader happy:
\begin{itemize}
\item Have a \texttt{README} file that (at least) describes what your
  software does, and which commands to run to obtain results. Also
  mention anything special that needs to be set up, such as
  toolboxes\footnote{For those who are
  particularly interested, other common structures can be found at
  \url{http://en.wikipedia.org/wiki/README} and
  \url{http://www.gnu.org/software/womb/gnits/}.}.
\item A list of authors and contributors can be included in a file
  called \texttt{AUTHORS}, acknowledging any help that you may have
  obtained. For small projects, this information is often also
  included in the \texttt{README}.
\item Use meaningful filenames, and not \texttt{temp1.m},
  \texttt{temp2.m}. The code should also unzip into a sub-directory.
\item Document your code. Each file should at least have a short
  description about its reason for existence. Non obvious steps in the
  code should be commented.
\item Describe how the results presented in your paper can potentially
  be reproduced.
\end{itemize}