
\section{Discussion (Computational Intelligence Laboratory Requirements)}
\label{sec:discussion}
\label{sec:cil}

Your semester project is a group effort. It consists of four parts:
\begin{enumerate}
\item The programming assignments you solve during the semester.
\item Developing a novel solution for one of the assignments, e.g. by
  combining methods from previous programming assignments into a novel
  solution.
\item Comparing your novel solution to previous assignments.
\item Writing up your findings in a short scientific paper.
\end{enumerate}

\subsection{Developing a Novel Solution}

As your final programming assignment, you develop a novel solution to
one of the four application problems. You are free to exploit any idea
you have, provided it is not identical to any other group submission
or existing Matlab implementation of an algorithm on the
internet\footnote{\url{http://www.ethz.ch/students/semester/plagiarism_s_en.pdf}}.

Two examples for developing a novel solution:
\begin{itemize}
\item You implemented a collaborative filtering algorithm based on
  dimension reduction as part of an assignment. Now you apply
  dimension reduction to inpainting.
\item You implemented both a clustering and a sparse coding algorithm
  for image compression. Now you combine both techniques into a novel
  compression method.
\end{itemize}

\subsection{Comparison to Baselines}

You compare your novel algorithm to \emph{at least two baseline
  algorithms}. For the baselines, you can use the implementations you
developed as part of the programming assignments.


\subsection{Write Up}

The submission must be in PDF form, using the \LaTeX{} template
corresponding to the IEEE style of publication. Refer to
Section~\ref{sec:latex-primer} for more information about preparing
your document. The document should be a maximum of {\bf 4 pages}.

\subsection{\LaTeX{} Primer}
\label{sec:latex-primer}

\LaTeX{} is one of the most commonly used document preparation systems
for scientific journals and conferences. It is based on the idea
that authors should be able to focus on the content of what they are
writing without being distracted by its visual presentation.
The source of this file can be used as a starting point for how to use
the different commands in \LaTeX{}. We are using an IEEE style for
this course.

\subsubsection{Installation}

There are various different packages available for processing \LaTeX{}
documents.
On Windows, use the Mik\TeX{} package (\url{http://miktex.org/}), and
on OSX use MacTeX
(\url{http://www.tug.org/mactex/2009/}). Alternatively, on OSX, you
can install the \texttt{tetex} package via
Fink\footnote{\url{http://www.finkproject.org/}} or 
Macports\footnote{\url{http://www.macports.org/}}.

\subsubsection{Compiling \LaTeX{}}
Your directory should contain at least 4 files, in addition to image
files. Images should be in \texttt{.png}, \texttt{.jpg} or
\texttt{.pdf} format.
\begin{itemize}
\item IEEEtran.cls
\item IEEEtran.bst
\item groupXX-submission.tex
\item groupXX-literature.bib
\end{itemize}
Note that you should replace groupXX with your chosen group name.
Then, from the command line, type:
\begin{verbatim}
$ pdflatex groupXX-submission
$ bibtex groupXX-literature
$ pdflatex groupXX-submission
$ pdflatex groupXX-submission
\end{verbatim}
This should give you a PDF document \texttt{groupXX-submission.pdf}.

\subsubsection{Equations}

There are three types of equations available: inline equations, for
example $y=mx + c$, which appear in the text, unnumbered equations
$$y=mx + c,$$
which are presented on a line on its own, and numbered equations
\begin{equation}
  \label{eq:linear}
  y = mx + c
\end{equation}
which you can refer to at a later point (Equation~(\ref{eq:linear})).

\subsubsection{Tables and Figures}

Tables and figures are ``floating'' objects, which means that the text
can flow around it.
Note
that \texttt{figure*} and \texttt{table*} cause the corresponding
figure or table to span both columns.


\subsection{Grading}

There are two different types of grading criteria applied to your
project, with the corresponding weights shown in brackets.
\begin{description}
\item[Competitive] \ \\
  The following criteria is scored based on your rank
  in comparison with the rest of the class.
  \begin{itemize}
  \item time taken for computation (10\%)
  \item average rank for all other criteria relevant to the task, for
    example reconstruction error and sparsity (20\%)
  \end{itemize}
  The ranks will then be converted on a linear scale into a grade
  between 4 and 6.
\item[Non-competitive] \ \\
  The following criteria is scored based on an
  evaluation by the teaching assistants.
  \begin{itemize}
  \item quality of paper (30\%)
  \item quality of implementation (20\%)
  \item creativity of solution (20\%)
  \end{itemize}
\end{description}

\subsection{Submission System}

The deadline for submitting your project report is Friday, 22 June
2012.
You need to submit:
\begin{itemize}
\item PDF of paper.
\item Archive (\texttt{.tar.gz} or \texttt{.zip}) of software. Please
  do not forget to include author information in the source archive.
\end{itemize}

\textbf{Important:} Please check the submission instructions on the webpage 
as it is the most updated instructions. 

